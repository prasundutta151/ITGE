\documentclass{article}
% Uncomment the following line to allow the usage of graphics (.png, .jpg)
%\usepackage[pdftex]{graphicx}
% Allow the usage of utf8 characters
%\usepackage[utf8]{inputenc}
\def\HI{H~{\sc i}~}
% Start the document
\begin{document}

% Create a new 1st level heading
\section{Main Heading}
section{NGC~628 Data}
\label{data}
\citet{walter08} have observed the HI emission from 34 spiral galaxies
using B, C and D array configurations of the Very Large Array (VLA) as
a part of The HI Nearby Galaxy Survey (THINGS). For our analysis, we
use the interferometric \HI ~data of the galaxy NGC~628 from
THINGS. NGC~628 is an almost face-on spiral galaxy with an average
inclination angle of $15^{\circ}$ at a distance of $7.3$ Mpc
\citep{deblok08}. The \HI~extent of the galaxy is $22.0^{'} \times
20.0^{'}$. We use the interferometric data with the primary
calibration from THINGS archive and use the Astronomical Image
Processing System (AIPS) for further analysis. We model the
synchrotron continuum of the galaxy using visibilities from the
channels without \HI~emission and make a continuum image. We perform a
few rounds of self-calibration to improve the signal to noise in the
continuum image and then do a continuum subtraction using the task
UVSUB in AIPS to retain only the \HI~emission in the visibilities for
further analysis. Figure \ref {fig:fig5} shows the moment0 \HI~map of
the galaxy NGC~628.

Figure \ref {fig:fig5} shows the estimated $C_{\ell}$ using I-TGE for
different windows. The left panel shows the measured $C_{\ell}$ when
we use a BW window with $\theta_b=7.5^{'}$ and $N=4$ (green curve in
Figure \ref {fig:fig1}). In this case, we aim to measure the
$C_{\ell}$ due to \HI for the whole region of the galaxy. The lower
range of $\ell$ will be affected by the convolution with the effective
primary beam (for details Figure 3 in \citet{samir14}). At large
multipoles, the \HI~signal falls very rapidly, and the $C_{\ell}$
estimated in this region is mainly noise dominated. We identify a
region in the $\ell$-space which we expect to be dominated by the
galactic \HI~ and fit a power law to the measured $C_{\ell}$. The
$\ell$ range which we have used for fitting is
$6\times10^3\le\ell\le6\times10^4$. The best fit value of the power
law index is $\beta=1.8\pm0.1$. The measured $\beta$ is roughly
consistent with earlier measurement by \citet{dutta13na} where they
got $\beta=1.6\pm0.1$. As discussed in \citet{dutta13na}, we can
interpret the power-law nature of the $C_{\ell}$ is due to the
two-dimensional ISM turbulence in the plane of galaxy's disk. The
middle panel of this figure shows the $C_{\ell}$ estimated using a BW
with smaller HWHM $\theta_b=3.5^{'}$ and $N=4$. Here, we want to
measure the $C_{\ell}$ from the inner region of the galaxy. We fit a
power law to the measured $C_{\ell}$ within the same $\ell$ range
mentioned above. In this case the best fit value of the power law
index is $\beta=1.7\pm0.1$. The right panel shows the same but for an
annulus window with inner radius $3.5^{'}$ and outer radius
$7.5^{'}$. Here, we aim to measure the fluctuations only from the
outer region of the galaxy. The best fit value of the power law index
is $\beta=1.89\pm0.1$. We did not see any significant difference in
the value of $\beta$ for the different region in this galaxy. We may
conclude that the power law index does not have any dependence on the
star formation rate within the galaxy. \citet{dutta13na} also found no
correlation between the power law index and star formation rate for
$18$ THINGS galaxies.


\section{Summary and conclusions}
\label{summ}
Statistical measurement in terms of power spectrum is an important
task in low-frequency radio interferometric observations. Earlier, we
have developed a visibility based Tapered Gridded Estimator (TGE) for
this purpose. The main features of this estimator are (a) it uses the
visibilities after gridding in ``uv'' plane to reduce the computation,
(b) it suppresses the contribution of strong sources located at the
outer edge of the primary beam, and (c) it calculates the noise bias
internally and subtracts this out to give an unbiased estimate of the
power spectrum. In this paper, we transform the visibility-based TGE
into the image-based TGE where the tapering is implemented in image
plane through multiplication. Earlier in visibility-based approach, we
apply the tapering in Fourier plane through the convolution. The main
advantage in the image plane is that we can use any taper window for
suppression although the analytical expression for its Fourier
transform is not known. Also, for small taper window function, the
correlation length in visibility plane will be large, and it will take
huge computation time to convolve all the visibilities in Fourier
plane. In that case, the image-based approach will be much easier to
apply the tapering during angular power spectrum estimation. The
image-based technique can also be applied to quantify the statistical
properties of a particular region of a galaxy (e.g. star-forming
region) and to mask a portion around strong sources in a radio image
which is affected by deconvolution errors. In this paper, we present
the mathematical formalism of this image-based TGE and validate this
using realistic simulations.

We validate the estimator using different taper windows
(e.g. Gaussian, Butterworth, Annulus and Mask windows). To simulate
the visibilities, we use realistic antenna distribution of VLA array
at $1.4{\rm GHz}$. We see that for all cases, except for the sharp
cutoff window, we are able to recover the model angular power spectrum
quite accurately. We see that the fractional deviation is less than
$10$ percent for $\ell\ge10^3$. But for the BW window with $N=256$
(Figure \ref {fig:fig1}), we see that there is a large deviation
between the estimated and model $C_{\ell}$. In this case, the sharp
cutoff in the image plane will create oscillations in the Fourier
plane. We expect the large deviation as compared to model is due to
this abrupt change of the window function at higher values of $N$. The
annulus window introduced here can be used to measure the statistical
properties only from the region we are interested in
(e.g. star-forming region in a galaxy). The mask window can be used to
avoid a corrupted region in the radio image caused by the
deconvolution error around strong sources.

We apply the I-TGE to the real galaxy NGC~628 data taken from THINGS
survey. We estimate the angular power spectrum $C_{\ell}$ at
different parts of this galaxy using different windows (e.g. inner,
outer and whole region of the galaxy). We identify a region in $\ell$
space which is likely to be dominated by the galactic
\HI~$(6\times10^3\le\ell\le6\times10^4)$. For all cases, we fit a
power law to the measured $C_{\ell}$ and found the almost same value
of the power law index $(\beta)$. The values we get for $\beta$ is in
the range $1.7-1.9$. This value of $\beta$ is consistent with earlier
measurement. We can interpret this result as fluctuations due to
two-dimensional turbulence in the galaxy plane. The similar values of
the $\beta$ at different parts of the galaxy signify that there is no
direct correlation between the $\beta$ values and the star-formation
of this galaxy. We plan to study more number of galaxies to give some
conclusion in this regard.

The I-TGE allows us to select any particular region in the image for
the angular power spectrum estimation $C_{\ell}$. However, it is
necessary to note that for a smaller window, the correlation length in
Fourier plane will be large and the Fourier mode at different bin will
be correlated. As a result, the covariance between the different bins
will be non-zero. In this paper, we only consider the variance at a
particular bin assuming different bins are independent. We plan to
study the covariance in future. Here, we only consider the angular
power spectrum which measures the fluctuations only in the
two-dimensional sky plane. It is also necessary to measure the
fluctuations along the line of sight in order to quantify the
cosmological $21-{\rm cm}$ signal. We plan to extend this formalism in
three dimensions to measure the power s
Your text goes here...

% Uncomment the following two lines if you want to have a bibliography
%\bibliographystyle{alpha}
%\bibliography{document}

\end{document}